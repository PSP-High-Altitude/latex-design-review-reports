\section{Avionics} \label{section:avionics}
\subsection{State Estimation and Apogee Determination}

Considering the team’s overarching goal of reaching the altitude of 100 km, an important responsibility of the avionics system is to verify the apogee reached by the vehicle. The objective of the flight will not be achieved without high confidence that the vehicle has reached the target altitude. Being able to verify the rocket’s apogee (\hyperlink{AVI.1}{AVI.1}) will be achieved through a culmination of different data in order to account for the highly dynamic environment throughout the spaceshot. During the flight, the avionics system will record several different datastreams, and an algorithm will combine all the data collected to estimate the state of the rocket (\hyperlink{AVI.1.1}{AVI.1.1}). This data will be stored onboard and recovered to allow a more accurate apogee verification after the flight (\hyperlink{AVI.1.2}{AVI.1.2}).


\subsubsection{System Architecture}
The avionics system will consist of both Commercial Off-The-Shelf (COTS) and Student Researched and Developed (SRAD) components. There will be both a primary and backup flight computer for redundancy on the vehicle. The current plan is to use the SRAD flight computer as the primary flight controller, with one or more COTS computers as backups. This setup is preferred because the SRAD computer allows for more customization of hardware and software to better match the overall profile of the flight. The SRAD computer will gather and store data from multiple sensors, estimate the rocket’s current state, trigger staging and recovery events, and possibly transmit data to the ground. It will also allow all the raw sensor data to be stored for analysis on the ground. However, extensive development and testing of the SRAD flight computer is still in progress. It may be determined after testing that higher customization of the SRAD board does not outweight the relaibility and flight heritage of commercially available options. In this case, the SRAD system will primarily record data for post-flight analysis, while a COTS computer will be the primary controller for the flight.

One COTS flight computer option under consideration is the AltusMetrum Telemega\footnote{The Altus Metrum Owner's Manual can be found at \href{https://altusmetrum.org/AltOS/doc/altusmetrum.pdf}{this link}.}. This computer includes a GPS, Inertial Measurement Unit (IMU), and barometer.  It has the ability to trigger staging and recovery events, based on altitude and attitude data, only triggering flight events when the programmed conditions are met. It also can transmit telemetry to the ground. However, it does have limitations. Due to COCOM regulations\footnote{Due to government regulations, commercially available GPS units will not provide data while they are above 60,000 ft (18 km) MSL and traveling faster than 1000 knots (500 m/s). Some manufacturers enforce both limits, while others only use one, and some use the speed limit with varying height limits.}, its GPS will provide data at altitudes greater than 50 km or velocities greater than 500 m/s. While this computer has Kalman filtering and could work on a 100 km flight, it has many limitations, as is further discussed in \Cref{section:data-fusion}. Finally, its telemetry downlink antenna only has a range of about 12 km, which is much less than the distance traveled throughout the flight.

By using the SRAD flight computer, GPS data could be obtained up to 80 km\footnote{The ublox MAX-M10S is capable of this altitude, per its \href{https://content.u-blox.com/sites/default/files/MAX-M10S_DataSheet_UBX-20035208.pdf}{data sheet}.}, and a more precise accelerometer and barometer could be used. It would also have custom software to match the mission of the rocket. This includes more specific conditions for triggering events, an improved Kalman filter that uses the predicted mass and aerodynamic profile of the rocket, and storing all the raw sensor data and GPS satellite pings for post-flight analysis.

The discussion until now has been about the avionics system on the second stage. If the first stage of the vehicle is expended, it will require no avionics system, but if is recovered, it will have its own avionics bay, which would likely be entirely COTS components to reduce complexity. A COTS system onboard the first stage would serve the simple purpose of parachute deployment. Because of the lower altitude, flight speed, and reduced flight complexity, an SRAD system would be unnecessary, and only serve to increase complexity and chance of failure. Whether or not the first stage is expended, the stage separation and ignition will be controlled by the second stage avionics system. Therefore, regardless of which design is used for the space shot, the design of the second stage avionics system will be unaffected.


\subsubsection{Data Fusion Algorithm} \label{section:data-fusion}

In order to provide telemetry and trigger flight events, the SRAD flight computer must know the current state of the rocket throughout the flight. ``State'' includes the rocket’s position, velocity, and attitude on all 3 axes. To do this, the computer will take measurements from three different sensors: a 6-axis IMU, a barometric pressure sensor, and a GPS. These sensors all have their own benefits and limitations, and their individual reliability will change throughout the flight. While GPS might provide the most accurate measurement of position, it will not be available for the entirety of the flight. Due to COCOM regulations, the GPS receiver will not provide data when moving faster than 500 m/s and/or above a set altitude, dependent on the specific module. Even without these restrictions, a GPS may have trouble obtaining a lock during the highest-speed portions of the flight. Additionally, there could be radio frequency (RF) interference from parts of the rocket, limiting the GPS signal transmission. We are aware of rockets operating in similar flight regimes experiencing these issues; USC's Traveler IV spaceshot did not have a GPS lock for the majority of the flight, and Derek Deville's Qu8k, despite carrying four different GPS modules, failed to maintain a lock.

The other two data streams are flawed as well. A pressure sensor, sampling the ambient air pressure, can be used to estimate the rocket’s current altitude. This requires adding holes to the airframe to equalize the pressure in the avionics bay with the external ambient pressure. However, shock waves from traveling at supersonic speeds can interfere with the ambient pressure readings. Additionally, pressure readings will become ineffective once the atmospheric pressure decreases below 10 mbar (\~25-30 km MSL). The 6-axis IMU measures acceleration and rotational rates. This data will need to be integrated to get a measurement of state, and this causes error to build up over time. Finally, all of these sensor measurements will have some level of noise which will decrease their precision.

In order to obtain the best state estimate from multiple measurement sources, a sensor fusion algorithm is needed. This sensor fusion algorithm will be some version of a Kalman filter, likely an Extended Kalman Filter (EKF). A Kalman filter combines a measurement with a predicted value, and outputs the estimated state along with a measurement of the uncertainty. An EKF works for non-linear systems, which include systems with multiple measurement inputs.  Although some COTS computers (such as the Telemega) use Kalman filtering, they are limited since they do not have any information about the flight profile or the aerodynamic characteristics of the rocket, so they fall back on a simpler physical model. An EKF run on the SRAD computer can provide a more accurate estimate, because the state update function can account for the thrust, mass, and drag profile of the rocket. This sensor fusion algorithm is currently being developed and tested using MATLAB and Simulink, as discussed in the testing section.

In addition to the onboard state estimation, the raw sensor data and GPS satellite pings from the flight will be saved for post-flight analysis on the ground. The data will be analyzed with more accurate methods that may be too computationally expensive to run during the flight. This will allow the apogee to be estimated with more precision.


\subsubsection{Testing} \label{section:avionics-testing}
The flight computer’s software algorithms are currently being developed in MATLAB Simulink, and will go through software in the loop testing in Simulink to validate them prior to flight. First, the 6DOF simulation will provide a theoretical flight path of the rocket. State data from this flight path is then converted into sensor data that would be generated by the computer’s sensors. This includes adding noise to the sensor outputs and matching the frequency and range of the sensor. This simulated data is then sent to the Simulink flight computer prototype. The flight computer’s state output can then be compared to the simulation output to judge the accuracy of the state estimation algorithm. This allows changes to be made to the algorithm to find optimize the estimate. This method will also be used to test the flight logic. Many potential failure modes can be tested to verify that the flight computer makes the correct decision in each scenario.

For verifying firmware functionality, we will unit test all sensor and hardware drivers extensively using fakes, which are essentially simplified software models of the corresponding devices. Unit tests will be written both for native desktop hardware as well as the target boards to balance ease of development and accuracy. Unit tests will use a hardware abstraction layer that calls into faked models of the hardware to allow seamless testing without hardware being available. We are also currently investigating doing hardware in the loop testing. However, due to the complexity of implementing such a system and the diminishing increase in failure scenarios being tested, we intend to pursue this only if we have extra time, and will not make this a limiting goal on our timeline.

Finally, the flight computer will be tested as a passive data recorder on many flights before it is the primary computer on any rocket. Its state estimation will then be compared to data from other COTS flight computers on the rocket to determine its accuracy.


\subsection{In-Flight Events}
As previously discussed, the spaceshot vehicle will have two stages with a recoverable upper stage. During the flight, this requires a stage separation and ignition followed later by a recovery system deployment. The avionics system will be in charge of sending the activations for these flight events (\hyperlink{AVI.2}{AVI.2} and \hyperlink{AVI.3}{AVI.3}). The activation conditions of these events need to be carefully verified before the rocket is permitted to proceed to the next stage of flight (\hyperlink{AVI.2.1}{AVI.2.1}, \hyperlink{AVI.3.1}{AVI.3.1}, \hyperlink{MEC.3}{MEC.3}, and \hyperlink{PRO.1.2.1.2}{PRO.1.2.1.2}), and this process will be completed by the avionics' onboard state estimation algorithm. After this verification occurs, the signal will be sent to activate the relevant subsystems on the rocket (\hyperlink{AVI.2.2}{AVI.2.2} and \hyperlink{AVI.3.2}{AVI.3.2}).

In order to ensure the flight events happen in the correct order, the SRAD flight computer will keep track of the current phase of flight. These phases can only happen in the intended order. Transitions between phases will be triggered when conditions using the rocket's position or acceleration data are met. These phases include: on the pad; first stage burn; first stage coast; second stage burn; second stage coast; and descent. There are specific conditions that must be met for transitioning between phases and triggering events which are based on altitude, acceleration, and/or attitude. For example, the second stage can only be ignited when the rocket is pointing less than a set angle from vertical. If this condition isn't met, there is an option for a second stage abort scenario. A more detailed summary and flowchart of phases and transition criteria is available in \Cref{section:avionics-appendix}. This flight sequence is modeled using Simulink and Stateflow, which can be compiled into C and run on the flight computer.


\subsubsection{Safety}
To ensure the safety of everyone working on the rocket, we intend to employ several subsystems that will ensure that energetic materials are not inadvertently triggered in unsafe situations or before they are expected to.

In order to prevent triggering of the motor or recovery hardware while the rocket is assembled, all avionics hardware will have a hardware power cutoff controlled by switches that will not be enabled until the rocket is vertical. We are currently exploring two primary cutoff schemes. The first is a WiFi switch, which can be opened and closed over a wireless connection. This is a device we have used in the past, and it has performed very reliably. However, the switch does not automatically give an indication of its state. A simple solution might be to wire a buzzer in-line. The second option that we are exploring is a mechanical pin, which arms the avionics when pulled out of the rocket. This might require a larger mechanism, along with a hole in the airframe, but it is conceptually simpler than the WiFi switch, and it gives a clear visual indication of its state.

Two systems to arm the flight computer that we are not pursuing are magnetic switches and key switches. On previous launches, we have used magnetic switches, but they have shown consistent reliability issues. Key switches embedded in the body of the rocket come with many of the benefits of the pin, but there is more structural complexity. Additionally, if the keyway protrudes from the vehicle at all, there are aerodynamic concerns.

We also plan to include software or hardware lockout timers that will ensure that staging or recovery doesn’t occur during early stages of the flight where potentially anomalous sensor data is expected (high acceleration and transonic regimes). On SRAD flight computers and programmable COTS boards, we intend to use time based software lockouts as we believe that the primary points of failure are the state estimation algorithms, which a software lockout at the flight logic level should be able to mitigate.

Finally, a similar system will be used to disable send stage motor ignition after a certain time period. This is to ensure that in the case of a second stage abort, it is safe to later approach and recover the rocket without fear of the second stage motor (which would still be loaded) accidentally being ignited. To minimize complexity, we intend to have an electromechanical system trigger a digital timer on launch that then cuts off a relay or transistor after the prescribed time (i.e. the system is prevented from igniting the second stage motor after five minutes have passed beyond the detected launch).


\subsection{Downlink}
It is critical that the team is able to locate the rocket after the flight is completed and the rocket has touched down (\hyperlink{AVI.4}{AVI.4}). In order to achieve this, the rocket will use GPS data so the recovery crew can have accurate coordinates of the landed vehicle (\hyperlink{AVI.4.1}{AVI.4.1}). Due to potential GPS limitations, the rocket will transmit the GPS data throughout the flight as it is received (\hyperlink{AVI.4.2}{AVI.4.2} and \hyperlink{STR.7}{STR.7}). This will allow the recovery crew to have accurate GPS readings of position and velocity for as long as possible, being able to estimate the final ground location if needed.

While it would be ideal to have live telemetry throughout the entire flight, it is most important to have a signal on the descent portion so that the rocket can be tracked for recovery. One significant challenge of live telemetry is getting a RF signal out of the airframe, which is likely to be metal. To solve this problem, the avionics bay will be inside an RF transparent section of the airframe. The length of this section of airframe will likely need to be at least the length of any antennas inside of it. This will also allow the GPS receivers to receive GPS signals. Even with this section, there may be other sources of interference to the antennas during flight. Also, the orientation of the antennas may affect their signal range. To ensure a strong signal on the descent, an antenna may be attached to the parachute or shock cord so that it leaves the body of the rocket after parachute deployment.

There are multiple COTS GPS modules that can transmit telemetry to the ground. One option is the Multitronix Kate-3 GPS Tracking System. This system apparently does not have a GPS altitude limit, but it does still have the COCOM speed limit of 500 m/s. It also includes an accelerometer, but this has a limit of 50G, which may be less than the forces experienced during flight. Because of this and its lack of a barometer, it is currently not planned to be used for triggering staging and recovery events. However, it may be used as the primary telemetry system since it can transmit telemetry to a range of 150 km.

On the other hand, an SRAD solution could be more adaptable to our specific needs. On the rocket, we could transmit telemetry using simple FM transmitter modules like the popular Radio\-metrix HX1 (but likely at higher frequencies) used in high-altitude balloons, or higher power XBee modules. We could experiment and research a wider range of frequencies, protocols, and data rates in order to find a solution that maximizes range and reliability. This would also give us the options to test different antennas to maximize range while decreasing space use.  Antennas will need further research and guidance as they have many tradeoffs, especially in the rocket. We want to try to maximize range, but the rocket’s antennas should also be omnidirectional to allow reception in any position the rocket may land. Additionally, we should try to limit the size of the RF transmissible airframe section, since it will decrease the overall structural integrity of the vehicle. On the ground we could use a combination of omnidirectional and directional antennas to allow flexibility in how we track the rocket. This could include a simple custom ground station with a software defined radio (SDR). This custom ground station would make it easier to use different antennas and also allow us to tune into different frequencies.


\subsection{Payload}
To satisfy our stakeholder requirements (\hyperlink{SR.5}{SR.5}), the spaceshot vehicle will have a payload, which will not be essential to the successful flight of the vehicle. The payload will include a camera, but if there is more available mass and volume,  we hope to include additional items. To support additional payload mass, as well as general overruns in component design, the vehicle is being sized with an apogee of 150km, well above our true target.

The camera system will, at a minimum, consist of a single camera looking radially out of the second stage. This imposes a requirement on the vehicle to de-spin if it is spin stabilized, so that good imagery can be captured (\hyperlink{MEC.1}{MEC.1}). The detailed design of the camera bay is beyond the scope of this review, but we expect the hole cut in the rocket to remain uncovered, as opposed to being blocked by a transparent window. The specific model of camera to be used will be based primarily on reliability and flight heritage; based on a brief study of comparable-performance amateur rockets, GoPro cameras seem to be the leading candidate.

If the payload subsystem is allocated more mass and volume than a single camera requires, we plan to add additional components to the payload. Ideas under consideration include

\begin{itemize}
    \item A camera inside the recovery bay, watching the deployment of the parachute
    \item A thermal camera inside the nosecone, to characterize the thermal loading
    \item A collection of COTS avionics boards, so we can later publish their performance on such an extreme flight
    \item A biological experiment, as minimal as a Petri dish, to explore the effects of a zero-g environment
    \item A LEGO Minifigure of the Star Wars character Mace Windu, which has flown on all previous HA flights
\end{itemize}

The specific components of the payload subsystem will be determined by PDR, once the actual mass and volume constraints are solidified.


\subsection{Durability}
While actively collecting, calculating, and transmitting data, the rocket will undergo very high accelerations and pressures. The avionics systems need to be able to survive these intensities throughout the flight with a successful recovery (\hyperlink{AVI.5}{AVI.5}). To ensure this survivability, the system will be prepared to withstand certain forces, vibrations, and thermal environments (\hyperlink{AVI.5.1}{AVI.5.1}, \hyperlink{AVI.5.2}{AVI.5.2}, and \hyperlink{AVI.5.3}{AVI.5.3}).

For the design of the SRAD board, we do not expect to have major problems with structural loading, since soldered and especially surface mounted (SMT) components should be able to handle forces well beyond what the rocket is expected to experience. However, we are considering potting the SRAD board to ensure any larger components like capacitors do not break off. We need to further investigate the viability and benefit of doing this. One potential issue with potting the board is that heat produced by the board may be trapped, leading to it overheating. However, we expect the power draw of our board to be low, so this issue is unlikely to be a problem. Another concern is that the barometer may get blocked from reading the ambient pressure. This issue requires more investigation into the specifics of potting.

For testing performance under acceleration, we are planning a test in which we will attach the board to the wheel of a vehicle, and drive it to simulate a ~200g load on the board (the maximum our planned linear accelerometer would be able to measure). This acceleration is likely much higher than any sustained forces during the flight, but there should be little to no cost to additionally verifying that the board can work up to those accelerations.

The biggest problem we anticipate is vibration causing breakage of connectors and wires going to and from the board. Our current plan of mitigating this is soldering as many connections as possible before assembly, and using latching connectors for any final connections. In addition, for the final flight we will try to use SMT components as much as possible, including for the storage media. We currently do not have a good way of testing the impact of this problem or the efficacy of our solutions, but we will continue to look into leveraging resources at Purdue and beyond to determine if we can perform useful vibration testing.

The final consideration related to avionics durability is thermal resilience. Many of the components we will be using have a maximum temperature rating as low as 85 degrees Celsius. Early thermal estimates show internal air temperatures in the avionics bay exceeding that value. It is not possible for us to take any action on this front until the structural design of the vehicle and the thermal analyses become more developed.

