\section{Avionics}
\subsection{State Estimation and Apogee Determination}
\subsection{In-Flight Events}
\subsubsection{Safety}
\subsection{Downlink}
\subsection{Testing}
\subsection{Payload}
To satisfy our stakeholder requirements (SR.5), the spaceshot vehicle will have a payload, which will not be essential to the successful flight of the vehicle. The payload will include a camera, but if there is more available mass and volume,  we hope to include additional items. To support additional payload mass, as well as general overruns in component design, the vehicle is being sized with an apogee of 150km, well above our true target.

The camera system will, at a minimum, consist of a single camera looking radially out of the second stage. The detailed design of the camera bay is beyond the scope of this review, but we expect the hole cut in the rocket to remain uncovered, as opposed to being blocked by a transparent window. The specific model of camera to be used will be based primarily on reliability and flight heritage; based on a brief study of comparable-performance amateur rockets, GoPro cameras seem to be the leading candidate.

If the payload subsystem is allocated more mass and volume than a single camera requires, we plan to add additional components to the payload. Ideas under consideration include

\begin{itemize}
    \item A camera inside the recovery bay, watching the deployment of the parachute
    \item A thermal camera inside the nosecone, to characterize the thermal loading
    \item A collection of COTS avionics boards, so we can later publish their performance on such an extreme flight
    \item A biological experiment, as minimal as a Petri dish, to explore the effects of a zero-g environment
\end{itemize}

The specific components of the payload subsystem will be determined by PDR, once the actual mass and volume constraints are solidified.

\subsection{Durability}