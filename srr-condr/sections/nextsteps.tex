\section{Next Steps} \label{section:nextsteps}
\subsection{Highest Risks}


% prop
The highest program risk related to the propulsion subsystem is currently the duration of the process of designing a new test stand and getting it approved by Zucrow personel. In the past, as part of propulsion design reviews with Zucrow, we have gone through several design iterations of our test stand. Designing a new test stand, especially if designed in conjunction with Zucrow to replace the broken test stand in the T-cell, means another developmental cycle that needs to be compliant with Zucrow's standards. This could set back testing full scale. Both stages on the spaceshot vehicle will need powerful solid motors that are not usually tested at Zucrow so getting approval for designs will not be as straightforward as we might like. Additionally, the second stage ignition cannot be adequately tested in a high-Mach regime during ground or subscale trials. The true conditions will be experienced for the first time on the spaceshot launch. 


% recovery
The highest risk items within the recovery systems consist of potential designs for both the sombrero (first stage) and reefing (second stage) parachute schemes. Neither of these parachute designs can be adequately tested during ground or subscale trials, meaning they will only experience the conditions presented by spaceshot on the final launch day. This makes it difficult to determine proper descent speeds across the significant distance from which the stages will be falling. This in turn makes the prediction of landing locations for both stages incredibly difficult. Also, since neither of these designs are widely used, there is a nonzero chance that both systems will have to be manufactured in-house. This could lead to significant challenges as extremely precise manufacturing is required involving parachute lines, something we have no prior experience with. These concerns could lead to prolonged manufacturing and testing times if not handled properly.

Primary risks associated with the separation system come mainly in the form of prolonged development time. The first prototype is currently being manufactured and will need to pass a number of tests before the design can be continued. Mainly, the mechanism must be able to hold atmospheric pressure in a near vacuum similar to flight conditions and burn at least 80\% of black powder detonated in vacuum separation tests. Failure for the system to achieve these goals would likely lead to the need for major redesigns and significantly prolonged development time.


% avionics
The primary risk surrounding avionics is the inability to completely verify the functionality of either an SRAD or COTS flight controller under the conditions they will be subjected to during a spaceshot. We were unable to find data on previous launches of the more popular COTS boards that go as high as this, which means that despite their proven track record at lower altitudes, they may not be as reliable when operating at these extremes. Furthermore, although we intend to verify the SRAD boards’ functionality with extensive modeling and unit testing, any full tests can only be done using simulation models that may or may not be accurate. On the hardware side of things, iteration is going to be expensive and time consuming due to needing custom PCBs. This problem is exacerbated by the ongoing chip supply issues, which may necessitate substitutions and thus redesigns over longer periods of time.



\subsection{Testing Capabilities}
The overall purpose of conducting testing is to prove that certain aspects of the system fulfill the outlined requirements. Tests are designed to return data showing that the part being tested is capable of the required functions for the overall rocket to function. During test operations, safety is of course our top priority, and we work with our advisors as well as the facilities we work at to develop and practice safe procedures.

Like any team, we do not have an unlimited capability to test our system. Often times, there are high-risk systems that we are not able to test with high fidelity, which is a difficult combination. This section is intended to summarize the capabilities that our team has for testing. As far as test flights, we are very capable of flights less than roughly 10,000 ft; these flights can be conducted locally. However, high-altitude test flights are very difficult for us, due to the distance to launch sites where such flights could occur. Our long-term plans for such tests are very much undetermined at the moment, but for the sake of practiciality, we hope to minimize the systems that can only be tested on such flights.

Our team is very lucky to work at Zucrow Labs, where we have the ability to manufacture and test motors. Hot-fire testing at Zucrow will be an important step as we approach our spaceshot attempt. We will be able to fully characterize the motors we will fly on, which will be included in our 6DOF to most accurately predict the parameters of the flight. Also, the better we can characterize the motor, the more accurate the state estimation algorithm running on the avionics board will be. While a simplified model of the motor in the avionics software would not likely lead to a mission failure, the better the motor model, the more accurate the in-flight estimation of state, and the more precisely the in-flight events can be triggered. An additional component of the propulsion system is the method by which the second stage motor will be ignited at a high altitude. A leading candidate to solve this problem is a burst disk. An upside of this design is the relative ease of testing it; an unloaded motor with a burst disk could be vacuum tested.

As was discussed earlier in \Cref{section:avionics-testing}, testing of the avionics system will develop over time. The core component of the avionics system, the state estimation algorithm, will receive software in the loop testing. Simulated flight data will be generated by the 6DOF, noise will be injected as the flight data will be passed through sensor models, and the state estimation algorithm will determine the parameters of the flight, which will be compared to the true data directly from the simulation. This will allow us to finely tune the parameters of the state estimation algorithm to perform will with flights like the spaceshot. We are also investigating hardware in the loop testing; however, it may be the case that the amount of time and effort required for this would outweigh the benefits. Finally, for all test flights, the avionics board will fly, at least passively. This will help us characterize the performance of the board and algorithm together, as well as reducing ``unknown unknowns''. The other siginificant component of the avionics system we will test is the downlink. Once transmitters, receievers, and antennas are selected, we plan to perform ground testing, bu simply moving the components far apart, as well as flight testing. However, these flights will not be very accurate in simulating the flight, particularly in terms of altitude, so more invesigation is needed.

We are relatively confident in our ability to test the mechanisms of the spaceshot vehicle. A draft of the separation mechanism is already being manufactured, with plans to fly in the spring semester. We are also able to perform higher-fidelity testing of this device on the ground. We are exploring testing the mechanism in a vacuum chamber to verify its performance during the spaceshot flight. The despin mechanism will also be somewhat straightforward to test on the ground with a spinning test rig. The parachutes and associated systems will be tested on lower-altitude test flights, which we are able to perform locally, and without significant overhead (as would be required for a high-altitude flight), though the conditions during these flights will not be able to replicate some aspects of the spaceshot launch. Unfortunately, the inter-stage mechanism will be very difficult to accurately test without a two-stage flight, which by their nature are more complex and risky than single-stage flights. The very simple design for the inter-stage we are pursuing helps to mitigate the lack of associated testing capabilities.

Testing of the aerostructures comes in two types: first, structural testing. We are confident in our abilities to characterize the structural characteristics of the vehicle through structural testing. Much more difficult, however, is testing the aerothermal loading the vehicle will experience. We are exploring testing options in this realm, but the design will likely need large safety factors for thermal loading, because the phenomenon is difficult to test, and to characterize in general.

\subsection{Timeline}
Next semester, we plan to make headway into the design of the vehicle. In particular, we plan to re-fly one of our existing vehicles (to roughly 5000 ft), primarily to test the stage separation mechanism, though the SRAD avionics board will be passively on board as well. However, we do not expect to hold another design review in the spring. Preparing for a design review stretches the limited resources of our team. We will work to balance the benefits a design review brings, especially in the form of feedback from reviewers, with the costs of holding one. For this reason, we plan to spend the spring semester entirely focused on design and testing, with our Preliminary Design Review tentatively expected to be held next fall. After that, we expect to hold a Critical Design Review, and finally a Flight Readiness Review, though naturally those reviews are much further out.
