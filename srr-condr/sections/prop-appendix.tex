\section{Propulsion System Safety Procedures} \label{section:prop-appendix}

\subsection{Example Procedure for Crawford Bomb Testing}

\subsubsection*{Pressurized Combustion of Energetic Materials in Crawford Bomb}
\begin{enumerate}
    \item NOTE: The ``two person rule'' applies to this procedure. A second researcher should be nearby and aware of experimental procedures being conducted. During sample preparation, other researchers may occupy the test cell while wearing appropriate PPE (safety glasses, nitrile gloves).
    \item During equipment setup and sample preparation, other researchers wearing
    appropriate PPE (safety glasses) may occupy test cell.
    \item Prior to conducting experiments, check the condition of the experiment (corrosion on
    ignition leads, integrity of windows, o-ring and gasket seals, and ensure window fixture
    bolts are tight). Replace or repair any damaged consumables prior to testing.
    \item Setup any diagnostics required (video camera(s), spectrometers(s), etc.).
    \item Turn on room draft fan (\(\sim\)50\% flowrate setting)
    \item Ensure electrical ignition power supply is off
    \item Place prepared energetic pellet in test article fixture
    \item Affix nickel-chromium ignition wire to energetic pellet and ignition wire leads
    \item Test electrical continuity of ignition circuit
    \item Tightly screw test fixture into the bottom of the combustion vessel
    \item Attach pressurant gas inlet tube to test fixture
    \item Attach electrical ignition leads to electrical feedthrough wires
    \item Prepare diagnostics for data acquisition trigger
    \item Evacuate personnel from the test cell and close test cell door
    \item Place ``Do Not Enter'' sign over test cell door knob
    \item Pressurize combustion vessel to 100 psi.
    \item Enter test cell to audibly check for gas leaks.
    \begin{enumerate}
        \item If a gas leak is identified, depressurize the vessel, re-enter, and tighten leaking fitting or seal.
    \end{enumerate}
    \item Evacuate personnel from the test cell and close test cell door
    \item Place ``Do Not Enter'' sign over test cell door knob.
    \item Pressurize combustion vessel to test pressure.
    \item Ignite energetic material by turning on and ramping power supply current.
    \item When combustion is complete, depressurize combustion vessel using solenoid actuated vent line.
    \begin{enumerate}
        \item If solenoid actuated vent line is clogged (slow depressurization), vent with the manual vent valve and replace vent line filters prior to next test.
    \end{enumerate}
    \item After venting to \(\sim\)100 psi, purge vessel with inert gas (\(\sim\)10 seconds) to exhaust combustion products.
    \item Depressurize vessel and re-enter test cell. Other personnel may also reenterthe test cell at this time
\end{enumerate}


\subsection{Example Procedure for Hot Fire Test}

\subsubsection*{Dual Propellant Pseudo Motor Procedures}

\subsubsection*{Fuel Grain Sample Preparation}
\begin{enumerate}
    \item WEAR Proper PPE (long pants, closed toed shoes, lab coat, safety goggles, nitrile gloves)
    \item GATHER driver grain and target grain sticks from propellant magazine
    \item CUT 5'' section from driver grain stick
    \item BORE a 1/4'' hole through the center of the driver grain with a drill bit by hand, ensure hole is centered
    \item MEASURE the maximum and minimum length and diameter of the driver grain (mm)
    \item WEIGH the driver grain (g)
    \item COAT both sides of the driver grain with 5 minute epoxy, and let cure for more than 10 hours
    \item APPLY masking tape to outer diameter of driver grain to match OD of driver grain with ID of driver grain pipe nipple
    \item CUT 1.25'' long section from target grain stick
    \item REMOVE cardboard insulator from target grain
    \item MEASURE the maximum and minimum length and diameter of the target grain (mm)
    \item WEIGH the target grain (g)
    \item STORE unused driver and target grains in propellant magazine, as well as prepared grains if not immediately running test operations
    \item DISPOSE properly of all waste and propellant shavings
\end{enumerate}

\subsubsection*{Test Area Setup}
\begin{enumerate}
    \item CLEAR test area
    \item OPEN garage door
    \item TURN ON NO\(_2\) sensor
    \item ASSEMBLE unistrut extensions to stand B in the T Cell
    \item ATTACH MI BNC cables to channel 1 and 3 of the oscilloscope. Attach trigger BNC cable to channel 4.
    \item TURN ON oscilloscope
    \item INPUT correct Oscilloscope settings:
    \begin{enumerate}
    \item Hit ``Acquire'' and ensure ``Mode'' is ``Hi-Res'' and ``Samples'' is ``1M''
    \item Ensure Channels 1, 3, and 4 are visible
    \item Ensure the termination of channels 1 and 3 is 50\(\Omega\) and channel 4 is 10M\(Omega\)
    \item Change the vertical resolution of channels 1 and 3 to 1mV and channel 4 to 10V
    \end{enumerate}
    \item SET UP microwave interferometery system in t-cell by positioning it on support material and attaching power brick and MI BNC cables
    \item CONNECT MI BNC cables to Microwave Interferometrer
    \item PLACE test warning signs outside t-cell in full view
    \item ENSURE Silver SCSI Cable which goes to the NI SCXI 1001 Chassis is connected to the back of the DAQ PC
    \item TURN ON DAQ PC
    \item TURN ON NI Chassis
    \item OPEN NI Measurment \& Automation Explorer. Open ``Devices and Interfaces'' -> NI PCIe-6351 -> NI SCXI-1001. Click ``Reset'' and ensure it was successful.
    \item START LabView data acquisition system
    \item OPEN main vi (Tcell\_main.vi)
    \item RUN main vi
    \item LOAD CONFIG -> Control Wiring -> DARPA Control
    \item LOAD CONFIG -> Data Wiring -> DARPA DAQ 3
    \item OPEN Schematics -> Darpa tmotor.vi
    \item START acquiring LabView data
    \item VERIFY power supply is turned OFF and ignition cables are disconnected from power supply and shunted
    \item VERIFY security cameras are turned on
    \item RECORD feed from camera channels 1, 3, and 4. Using the mouse hanging from a hook on the server rack, right click, hit ``Main Menu'', click the crossed hammer and screwdriver (2nd from the right), click ``Storage'', on the ``Record'' tab, ensure that 1, 3, and 4 are set to ``Manual''. Click OK and right click to remove menus
    \item RECORD ambient temperature (deg F). Use www.weather.gov w/ zipcode 47907
    \item RECORD atmospheric pressure (mb) and humidity (\%). Use www.weather.gov w/ zipcode 47907
\end{enumerate}

\subsubsection*{T-Motor Assembly}
\begin{enumerate}
    \item INSPECT all components for structural impurities
    \item CUT EPDM rubber to fit the shape of the cross fitting
    \item APPLY five minute epxoy to EPDM rubber, fit into the cross fitting, and let cure
    \item APPLY three wraps of Teflon tape to all male pipe threads
    \item INSERT driver grain into the long pipe nipple
    \item ATTACH cap to end of 5'' pipe nipple
    \item TIGHTEN cap to end of nipple a turn and a quarter greater than hand tight
    \item ATTACH driver grain assembly to cross fitting
    \item TIGHTEN driver grain assembly into cross fitting
    \item INSERT teflon cone into target grain plug
    \item APPLY five minute epoxy to target grain plug, and insert alumina disk
    \item INSERT target grain into target grain plug
    \item ATTACH target grain assembly into cross fitting
    \item TIGHTEN target grain assembly into cross fitting
    \item RECORD nozzle throat diameter (mm)
    \item INSTALL o-ring onto nozzle and lubricate with vacuum grease
    \item INSERT nozzle into 2.5'' pipe nipple
    \item ATTACH nozzle cap onto pipe nipple and ensure o-ring seat
    \item TIGHTEN nozzle cap
    \item ASSEMBLE burst disk
    \item TIGHTEN burst disk
    \item ATTACH 1/4 NPT tee-connector to instrumentation plug
    \item FILL 1/4 NPT pipe nipple with high temperature oil
    \item APPLY vacuum grease to pipe nipple
    \item ATTACH pressure transducer and burst disk to 1/4 NPT tee-connector
    \item TIGHTEN instrumentation to 1/4 NPT tee-connector
    \item ATTACH instrumentation plug to cross fitting
    \item TIGHTEN instrumentation plug to cross fitting
    \item THREAD two e-matches through nozzle throat
    \item CUT OFF finger from pair of nitrile gloves and fill with 0.25g of excess propellant shavings from sample prep
    \item ATTACH igniter bag to end of ematch and insert into port of driver grain
    \item APPLY nitrocelluose lacquer in a flat coat on the top of the target grain and attach the ematch
    \item ATTACH nozzle assembly to cross fitting across from driver grain assembly
    \item TIGHTEN nozzle assembly
    \item ATTACH instrumentation assembly to cross fitting across from target grain assembly
    \item TIGHTEN instrumentation assembly with pressure transducer facing towards driver grain assembly
    \item VERIFY all connections have been sufficiently tightened
    \item ALIGN T-motor assembly to mounting plate
    \item ATTACH T-motor assembly to mounting plate, sliding U bolts around the target grain and instrumentation assembly ports in the cross fitting
    \item TIGHTEN the U-bolts to the mounting plate
    \item ATTACH full T-motor assembly to unistrut stand with nozzle facing outside of test cell
    \item CONNECT pressure transducer to DAQ (AI-11)
    \item CONNECT teflon waveguide to target grain plug
\end{enumerate}

\subsubsection*{Run Test Operation}
\begin{enumerate}
    \item TURN ON microwave interferometry system
    \item CENTER signal from microwave interferometery on oscilloscope
    \item SET scale on oscilloscope to 40.0s
    \item VERIFY microwave interferometery is sending signal by running hand along teflon waveguide and observing large signal change on the oscilloscope
    \item TURN ON the system relay panel in the control room (Flip the large red switch on the front of the server rack)
    \item VERIFY proper readings from pressure transducer and load cell
    \item SET UP blast plate behind T-motor assembly to protect other hardware
    \item TURN ON fan in t-cell with exhaust facing outside of test cell
    \item VERIFY warning signs outside of test cell are still up
    \item PLACE signal LED in view of security camera
    \item ATTACH igniter leads to e-matches in parallel
    \item SET timer for five minutes
    \item PRESS ``single'' on oscilloscope. NOTE: AFTER THIS STEP MI DATA WILL ONLY BE COLLECTED FOR FIVE MINUTES. IF TIME TO TEST EXCEEDS FIVE MINUTES ABORT TEST OPERATIONS AND RETURN TO STEP 11
\end{enumerate}
\textbf{EVACUATE ALL PERSONNEL FROM T-CELL AND 116 BLAST ROOM BEFORE PROCEEDING}
\begin{enumerate}
    \setcounter{enumi}{13}
    \item WARN all non-test personnel in 110 and surrounding area of impending test
    \item TURN OFF air-conditioning unit in 116 blast room
    \item TURN ON overhead vent in 116 blast room to 100\%
    \item VERIFY test area is clear
    \item PLACE chains on 116 blast room, t-cell, and outside t-cell control room
    \item TURN ON outside and inside warning lights
    \item VERIFY personnel are accounted for
    \item VERIFY continuity in igniter with multimeter without leads being connected to power supply by flipping switch ``Flip this one'' in labview interface. Make sure the switch is set to OFF before continuing.
    \item ATTACH leads to power supply
    \item TURN ON power supply
    \item RECORD data
    \item SCAN area to verify all readings and states are as expected
    \item SOUND klaxon alarm for five seconds
    \item COUNT DOWN from five
    \item RUN test (press control ``Flip this one'' in labview interface)
\end{enumerate}

\subsubsection*{Shut Down, Data Collation, and Clean Up}
\begin{enumerate}
    \item STOP acquire
    \item TURN OFF power supply
    \item DISCONNECT leads from power supply
    \item VERIFY pressure and thrust data saved to TDMS file. Double click the TDMS file and save the excel document to the DARPA data folder on the desktop
    \item SAVE all video data to flash drive
    \item SHUT DOWN LabView system
    \item WAIT for T-motor assembly to cool down and exhaust smoke to clear from test area
    \item REMOVE chains from outside control room, 116 blast room, and test cell
    \item TURN ON air-conditioning unit in 116 blast room
    \item Turn OFF overhead vent in 116 blast room
    \item TURN OFF warning lights
    \item ENTER test cell
    \item DETACH teflon waveguide from T-motor assembly
    \item TURN OFF microwave interferometer
    \item WAIT until the the oscilloscope has recorded a full screen of data (5 minutes after single has been pushed)
    \item SAVE screenshot and signal data from oscilloscope to flash drive
    \item TURN OFF oscilloscope
    \item REMOVE test warning signs from outside test cell
    \item REMOVE oscilloscope, BNC cables, and microwave interferometery system from test cell
    \item DETACH AI cable from pressure transducer
    \item REMOVE T-motor assembly from unistrut, bring into control room
    \item REMOVE unistrut extensions from stand B
    \item CLOSE garage door
    \item CLOSE DOWN test cell
    \item DISASSEMBLE T-motor assembly
    \item CLEAN each part of T-motor assembly thoroughly with acetone and wipes
    \item PUT AWAY all T-motor components    
\end{enumerate}


\subsubsection*{Abort and Emergency Procedures}

\subsubsection*{Hang Fire}
\begin{enumerate}
    \item TURN OFF power supply
     \item REMOVE igniter leads from power supply
     \item TEST continuity of ignition system
     \item If continuity fails WAIT five minutes before entering test cell
     \item TEST continuity at all points in system
     \item If continuity fails again DISASSEMBLE T-motor assembly and REPLACE igniter, resume procedures at step 29 in ``T-Motor Assembly''
     \item If continuity does not fail REPLACE leads in control room
     \item If continuity is held RESUME procedures at step 13 in ``Run Test Operation''
\end{enumerate}


\subsubsection*{Rapid Unscheduled Disassembly}
\begin{enumerate}
    \item TURN OFF power supply
     \item REMOVE igniter leads from power supply
     \item LISTEN for NO\(_2\) alarm
     \item If NO\(_2\) alarm sounds EVACUATE building immediately and CALL Dr. Pourpoint at (765) 463-1615
     \item If alarm does not sound TURN ON personal NO\(_2\) sensor and WAIT for smoke to clear in test cell
     \item ENTER test cell
     \item INSPECT other setups in room for damage
     \item If other setups are damaged EVACUATE building immediately and CALL Dr. Pourpoint at (765) 463-1615
     \item If setups are not damaged DISCONNECT and REMOVE microwave interferometery system from test cell
     \item REMOVE T-motor assembly from test cell, bring into control room
     \item SHUT DOWN LabView system
     \item CLEAN and INSPECT T-motor assembly for damage
     \item INFORM Dr. Son or Tim Manship of anomaly
     \item SHUT DOWN test setup as per section five, ``Shut Down, Data Collation, and Clean Up''
\end{enumerate}


\subsubsection*{Power Outage}
\begin{enumerate}
    \item TURN OFF power supply
     \item REMOVE igniter leads from power supply
     \item WAIT for power to come back on
     \item If power comes on in less than ten minutes RESUME test operations
     \item If power does not come back on for more than ten minutes ENTER test cell
     \item REMOVE igniter leads from T-motor assembly
     \item REMOVE T-motor assembly from test cell, bring into control room
\end{enumerate}
